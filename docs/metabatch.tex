\documentclass[11pt,letterpaper,portrait]{article}
\usepackage{urtechdoc}
\usepackage{urfooter}

%%%%%%%%%%%%%%%%%%%%%%%%%%%%%%%
% Choose one of these typefaces instead of computer modern (the default).
%%%%%%%%%%%%%%%%%%%%%%%%%%%%%%%
%\usepackage{bera} % Clean text, formats well, photocopies well.
%\usepackage{yfonts} % Blackletter Fraktur typeface.
%\usepackage{libris} % Formats a little better than computer modern.
%\usepackage{tgschola} % Great support for extended character sets.
%\usepackage{venturisold} % Artful, old looking type.

\usepackage[T1]{fontenc}


\SetWatermarkText{DRAFT}
\SetWatermarkFontSize{5cm}
\SetWatermarkAngle{58}
\SetWatermarkColor[gray]{0.97}

\setlength{\parskip}{0.3em}
\setlength{\parindent}{0.8em}
\renewcommand{\baselinestretch}{1.0}
\providecommand{\MB}{{\fontfamily{pzc}\selectfont \Large M\textsc{eta}B\textsc{atch}}\xspace}

\title{\MB \\
\large Preliminary Design}
\author{George Flanagin}
\date{Revision of \today}

\begin{document}
\maketitle

\achtung{\MB is not a very good name. The recent collapse of the
Metaverse and generally bad feelings about the Meta corporation
have made \MB an unattractive name, even if it is descriptive. We
are currently soliciting better names from the user community.}

\section{What does \MB do?}

\MB has three functions:

\begin{enumerate}
\item \textbf{Optimized use of the cluster.} SLURM's scheduling
algorithm for any partition is a simple search for the next available
space on the next available node, rather than best fit. On clusters
with 1000 nodes that are generally very busy, the algorithm works
well enough. On smaller clusters, we want to use every available
cycle.

It we think of the cluster usage as a number of finite rectangles
into which we wish to place LEGOs of varying sizes, the ideal is
to find a place where a job will fit exactly, thereby using all
the cores on a node, and all its memory. 

Alternatively, we can override SLURM's behavior to always schedule
on lower numbered nodes before higher numbered ones within the same
partition.

\item \textbf{Application of \UR's business rules.} As an example,
there is no way to tell SLURM that the owner of a condo-node is on
sabbatical or that it is OK for a particular user to run jobs on a
node that the user does not ordinarily use without editing the
\lit{slurm.conf} file. The \lit{slurm.conf} file is designed
(primarily) to express facts about the hardware rather than apply
business rules.

\item Functional proofreading of SLURM scripts. For example, the
Gaussian software is not aided by more than 12 cores, and \MB can
edit SLURM files before they are submitted to be run.

\end{enumerate}

\section{User experience}

In general use, the users will have no experience of metabatch. As
they have always done, they will type \lit{sbatch somefile.slurm}
to submit jobs for scheduling. A system wide shell function named
\lit{sbatch} will wrap itself around the ``real'' \lit{sbatch}
command. Any non-default operations will be concealed inside ``expert
mode'' switches. At the moment, there does not seem to be an obvious
need for many options, perhaps only these three:

\begin{description}
\item[-{-}dry-run] Perform a syntax check and edit of the submitted
SLURM script, but do not run it, and instead tell the user what
would have happened. 

\item[-{-}no-best-fit] Let SLURM send it to the next available
location rather than searching for place to run that uses the node
maximally.

\item[-{-}test] Adjust the time to some short duration (named in
the config file mentioned below) for the purpose of assessing
resource usage.

\end{description} 

It is worth noting that, should \MB fail or crash, the \lit{sbatch} 
submission will revert to the standard SLURM program of the same
name and the operations it is configured for. 

\section{Design of the \MB program}

\begin{enumerate}
\item The primary code will be located in a resident \daemon,
\lit{metabatchd}.

\item Configuration will be in a multipart, standard configuration
file in \lit{/etc/meta\-batch\-/config}, and additional configuration
files in \lit{/etc/metabatch/conf.d}. 

The configuration file will contain information about:

\begin{itemize}
\item Business rules that involve users and condo-nodes.
\item Program limits for CPU and memory allocation.
\item Rules about load balancing across nodes.
\end{itemize}

\item Communication with the \daemon will be done by sending 
standard Linux signals. For example, standard use in Linux is
\lit{SIGHUP} for re-reading configuration files, \lit{SIGTERM} 
to close the program in a graceful shutdown, \etc

\item It is unnecessary for the \daemon to be running on the headnode.
The \daemon only needs to know the IP addresses of the headnode[s]. 
In this situation, one \daemon could conceivably carry out campus
wide scheduling. 

\item SLURM can provide all the information about instantaneous use
of the cluster and assignment of CPU and memory on each node. This 
allows for stateless optimization and no need to maintain a copy of
any configuration information.

\item SLURM itself is run as the privileged user, \lit{slurm}, on
our systems and most other systems. Even on \spydur where we lack
\lit{root} access, we have access to both running processes on 
behalf of the \lit{slurm} user and the \lit{installer} user, and
the latter can impersonate ordinary (``LDAP'') users.

\item \MB will use the feature in SLURM to assign processes to 
specific nodes. Users who submit jobs will continue to submit
the jobs to \emph{partitions}, so no changes to user behavior
will be required. 

\end{enumerate}

\EOD

\URfooter

\end{document}

